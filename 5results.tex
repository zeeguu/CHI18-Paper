%!TEX root=paper.tex


\input{5-personalization_being_used}

%!TEX root=paper.tex

\newpage
% \section{How Do Students Improve Their Vocabulary?}
% \section{How Does The System Help Students Improve Their Vocabulary?}
% \section{What Is The Impact of the System on the Learner Vocabulary?}
% \section{Does The Vocabulary of the Learners Improve?}
\section{Do Students Improve Their Vocabulary?}

  The value of extensive reading can be found, besides the new words that are learned, in the strengthening of the knowledge of the existing words, increased fluency, and increased grammar knowledge. Some of these benefits can be reliably measured only after a time longer than our deployment \cite{renadya07-power}. 

  However, since our system combines free reading with vocabulary exercises and tracks all word interactions, by analyzing the learner interaction with the reader and the exercises, we can provide a glimpse into two measures of progress visible after one month of usage: increasing confidence about words and learning new words. 

   \begin{figure}[h!]
  \centering
    \includegraphics[width=0.8\columnwidth,trim={0 10 0 15},clip]{figures/word-learning-flow.pdf}
    \caption{Word encounters in the Reader and Exercises}
    \label{fig:word_learning_flow}
  \end{figure}

  Figure \ref{fig:word_learning_flow} summarizes visually the interactions of the students with words in the reader and the exercises parts of the system. The figure is based on the analysis of the user interaction database and shows that: 

  \begin{itemize}
    \item From the 6,721 words that were looked up in the reader, 5,149 were used in the exercises platform \cite{Avagyan17a-blocks}. Since the learners requested a translation for these words we can assume that they were not known to the readers or at least the readers were unsure about their meaning. 

    \item More than 1,500 words were not practiced in the exercises. 
    Some did not get their turn to be scheduled by the algorithm and others were not presented because the system deemed them not fit for study (cf. Vocabulary Recommender, p.3). 
    % Given that not all the words can be practiced, this is an argument for prioritizing the words that the students are presented with in the exercises.

    \item For 4,111 words (80\% of all the words present in exercises) the learners were able to correctly identify the meaning in the last associated exercise. Out of these: 

    \begin{itemize}
      \item 720 words (14\% of all the words in exercises) were wrong during their first exercise interaction but were correct in the final one. These {\bf 720 words are likely to be learned via the exercises}\footnote{Likely because they might still be forgotten later} by the sixty students in our study. They represent 10.75\% of all the words that the students translated in the reader.

      \item 3,391 words (66\% of all the words in exercises) were recognized already for the first time in the exercises. These are {\bf likely to be words for which the knowledge was strengthened by using the system}: the students were unsure when encountering them initially in the reader but eventually recognized their meaning when encountering them later in the exercises\footnote{It could also be that the students learned them after the first encounter in the text, but we keep the more conservative hypothesis}. 
    \end{itemize}

  \item For 1,038 words (20\% of all the words present in exercises) the outcome of the final exercise that involved them showed an incorrect answer. Thus we can assume that they were {\bf still not learned at the end of the experimental period}.

  \end{itemize}

% \end{added}


% \newpage
% \section{How Are The Reader Features Used?}
\section{How Do Learners Interact With the Reader?}
% \section{What Is The Relative Importance of the Various Interaction Modes?}
\newcommand{\feature}[1]{{\em #1}}
The reader interaction is more innovative and complex than the exercises.
This is why we use telemetry to investigate how do learners use the features of the reader. 

Telemetry has been successfully used for understanding user behavior in games \cite{Gagne11-telemetry} but also more generic contexts, such as automatically detecting personas from large scale interaction data \cite{Zhang16-telemetry}. In our study, we used telemetry to track the usage of various relevant features in the reader of the personalized textbook in order to better understand the usage of our system.

Based on logging every interaction of every user, Figure \ref{fig:feature_usage} (left) shows the six most used features of the system.\footnote{An extended analysis that includes more features is elsewhere. \cite{Chirtoaca17-apollo}} Figure \ref{fig:feature_usage} (right) shows the number of distinct users for each category of events. A larger number of distinct users indicates a feature that is more important to the students. 

  \begin{figure}[h!]
  \centering
    \includegraphics[width=1.0\columnwidth]{figures/feature-usage}
    \caption{Popularity of features by their recorded usage-events (left) and number of users that use them at least once (right)}
    \label{fig:feature_usage}
  \end{figure}

% With 6.700 occurrences, 
\feature{Requesting a translation} is the most used interaction of the system and \feature{showing translation alternatives} is the second most used one. The six-to-one ratio between the two features (6,721 to 1,146 as in Figure \ref{fig:feature_usage}) is an indicator of the limitations of the automatic translation. The fact that they are both achievable with one click or touch proves to have been a good decision. 
 % -- these are probably the situations when learners ask for 

\feature{Pronuncing a word} is he third most used interaction. On average, there are about 1.66 pronunciations for a given translation, suggesting that users are often asking for a second pronunciation after hearing it the first time. 

% \ml{@dan, do you agree with this conclusion? it's opposite to your thesis, but I think this is the correct interpretation}

% In addition, we looked at the number of times the same word or phrase was pronounced by the same user. This data ranges from one single pronunciation to 14 pronunciations for the same word (phrase). The size of this interval is mostly due to the users' different proficiency in a certain language and the difficulty in pronunciation of the word (phrase) itself. Nevertheless, on average, the number approaches 1.66 pronunciation requests for the same piece of text, suggesting that users are generally sufficiently content with a pronunciation after hearing it the first time.


\feature{Undo-ing a translation} is used when the user wants to remove the last translation that was inserted in the text. For the proposed interaction mechanism this feature seems useful. 

\feature{Liking} an article that was just read by clicking the corresponding button at the bottom of an article happened 174 times. This information can be used in the future to improve article recommendations.
 % and maybe to add a social dimension to the system by providing information about how other people react to a given article.

\feature{Suggestion of an alternative} allows users to contribute their own translations when they are not satisfied with the one automatically provided by the system. This interaction is used seldom and by only a minority of users. It still is to be determined whether this is due to readers being satisfied with the automatic translations and their alternatives, or due to a low involvement. It might also be that more advanced readers would benefit more from this feature.
% \ml{which brings me to: @Dan, what do you mean by alternatives here? :) Is it the number of times somebody selected an alternative, or the number of times they opened the menu. In any case, can we get the other number?}

  % \begin{figure}[h!]
  % \centering
  %   \includegraphics[width=0.6\columnwidth]{figures/feature-usage}
  %   \caption{The usage of the various reader features by the various users }
  %   \label{fig:usage_per_user}
  % \end{figure}


%  We see that: 
% \begin{itemize}
%   % \item Not all the users of the system use translations
%   % \item \feature{Translation suggestion} is used by very few users. It still is to be determined whether this is due to readers overwhelmingly being satisfied with the automatic translations and their alternatives, or due to a low involvement. 
% \end{itemize}




\newpage
\section{How Do Students Interact With Exercises?}

The system presented four types of vocabulary practice exercises to the students. In total, during the entire duration of the study we observed 18,082 attempts being submitted by the students in 14,609 exercises\footnote{Attempts include wrong submissions, and requests for hints}. Figure \ref{fig:ex_interactions} presents the number of answers which had a ``correct'' outcome (red) vs. exercises which had a ``wrong'' outcome (blue). The figure shows one student who submitted 2,865 answers during one month, and about six eager students who submitted about  700 answers each. 

  \begin{figure}[h!]
  \centering
    \includegraphics[width=0.9\columnwidth]{figures/exercise_interactions_count.png}
    \caption{Correct (red) and wrong (blue) exercise outcomes per student}
    \label{fig:ex_interactions}
  \end{figure}

The figure does not include one other type of outcome, {\em requesting a hint}, which is presented in the table below grouped per exercise type. The corresponding number of hints suggests that the multiple-choice exercises (i.e. Match, Choose) are simpler than free text entry exercises (i.e. Find, Translate).

\begin{tabular}{lrrrr}
  % source id: 
  % choose -- 5
  % find -- 4
  % translate -- 7
  % match -- 6
                      & Choose  & Find & Translate & Match \\ \hline
  Total attempts  & 7,180    & 6,249 & 2,643      & 2,010\\
  Hint requests       & 29      & 529  & 847       & 16 \\ \hline
  \label{tab:hints_per_ex_type}
\end{tabular}

Figure \ref{fig:activity_per_day} shows the days when learners practice exercises. The x-axis has the days of June and the y-axis has the different user ids. The figure suggests that the students are doing exercises at their own pace over the observed period. The activity is rather sparse, with a more intensive period towards the end 

  \begin{figure}[h!]
  \centering
    \includegraphics[width=0.7\columnwidth]{figures/user_exercise_activity_vs_day.pdf}
    \caption{The students are doing exercises at their own pace throughout the one month interval }
    \label{fig:activity_per_day}
  \end{figure}


% % \begin{added}




% \newpage
% % \section{How Do Students Improve Their Vocabulary?}
% % \section{How Does The System Help Students Improve Their Vocabulary?}
% \section{What Is The Impact of the System on the Learner Vocabulary?}

%   The value of extensive reading and vocabulary practice can be found, besides the new words that are learned, in the strengthening of the knowledge of the existing words, increased fluency, and increased grammar knowledge. Some of these benefits can only be clearly measured after a longer time \cite{renadya07-power}. 

%   However, since our system combines free reading with vocabulary exercises, by analyzing the learner interaction with the reader and the exercises we can provide a glimpse into two measures of progress visible after one month of usage: {\em increasing confidence about words} and {\em learning new words}. 

%    \begin{figure}[h!]
%   \centering
%     \includegraphics[width=0.9\columnwidth]{figures/word-learning-flow.pdf}
%     \caption{An overview of the words encountered in the Reader and practiced in the Exercises}
%     \label{fig:word_learning_flow}
%   \end{figure}

%   Figure \ref{fig:word_learning_flow} summarizes visually the interactions of the students with words in the reader and the exercises parts of the system. The figure is based on the analysis of the user interaction database and shows that: 

%   \begin{itemize}
%     \item From the 6,721 words that were looked up in the reader, 5,149 were used in the exercises platform during the learning period. Since the learners requested a translation for these words we can assume that they were not known to the readers or at least the readers were unsure about their meaning. 

%     \item More than 1,500 words were not practiced in the exercises. Some of these never got the chance to be scheduled by the algorithm and others were not presented to the students because the system decided they were not interesting enough. 
%     % Given that not all the words can be practiced, this is an argument for prioritizing the words that the students are presented with in the exercises.

%     \item For 4,111 words (80\% of all the words present in exercises) the learners were able to correctly identify the meaning in the last associated exercise. Out of these: 

%     \begin{itemize}
%       \item 720 words (14\% of all the words in exercises) were wrong during their first exercise interaction but were correct in the final one. These {\bf 720 words are likely to be learned via the exercises} by the sixty students in our study. They represent 10.75\% of all the words that the students translated in the reader.

%       \item 3,391 words (66\% of all the words in exercises) were recognized already for the first time in the exercises. These are {\bf likely to be words for which the knowledge was strengthened by using the system}: the students were unsure when encountering them initially in the reader but eventually recognized their meaning when encountering them later in the exercises. 
%     \end{itemize}

%   \item For 1,038 words (20\% of all the words present in exercises) the outcome of the final exercise that involved them showed an incorrect answer. Thus we can assume that they were {\bf not learned at the end of the experimental period}.

%   \end{itemize}

% % \end{added}














