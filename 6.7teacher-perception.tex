%!TEX root=paper.tex

\newcommand{\q}[1]{($Q_#1$)}

\subsection{The Evaluation of the Teacher}
After the deployment with the school children was finished, we conducted a semi-structured interview with the teacher of the class to gain insight into his perception of the usefulness, benefits and limitations of the system. Overall, the conclusion is that the integration of the system in the three classes worked well. Some of the other ideas that stem from the interview are: 

\begin{itemize}

	\item Such a system is critical for language education in school, since the choice of topics is motivating for the students \q{3}\footnote{The $Q_n$ annotations refer to the questions in the full text of the interview which is available online at: \url{https://github.com/zeeguu-ecosystem/CHI18-Paper/blob/master/data/teacher-interview.txt}}. 

	\item The system should only be used for students who had already two or three years of prior foreign language experience \q{5}. 

	\item The sources that were used were mostly general. There was only one source with sports and the number of students mostly boys found this to be very interesting. Maybe more sources for other specific subjects would be good \q{7}.

	% \item usually I made them learn vocabulary by heart for a test but this has proven to be rather useless because it only improves their short-term memory.  the good thing about your system is that they encounter a lot of words a lot of times and also you have spaced repetition and that is very good for learning vocabulary

	\item There is no danger that every student will develop his little individual vocabulary bubble. The teacher believes that once the students have a solid basic vocabulary, it is perfectly acceptable that they study the words which interest them. Moreover, he believes that there might still be an overlap between the words studied by different learners due to the skewed distribution of word frequency \q{5}

	\item The fact that the translations are not perfect, and every now and then a student must find an alternative translation is ``more than acceptable''. This might have the students be more actively engaged with the texts that they are reading. 
	% If 98% of the translations are correct they only click on the world and they continue reading but when they look at the word and have to think about whether is the right translation or not, they are more active. You

	% 
	% students have been studying French for 3 years the most frequent vocabulary they already have. I am an experienced teacher and language scientist as well; I know that when you have this base vocabulary and you go beyond it most of the world that just in the text are still the same, a small proportion of the words the people use. If you look at the French language, there are 25’000 words but when you're reading an article, you only encounter very few of those 25’000 words. When students read different texts, I think in the end 75% of the words that they will learn will be the same. 

\end{itemize}


The teacher of the class appreciated the system, and decided to introduce it in the entire new academic year with a larger group of students. 
% However, not all the new classes are bilingual, we will have to explore the best approach for situations in which the L1 is not English but Dutch.
